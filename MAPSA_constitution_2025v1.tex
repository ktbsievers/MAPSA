\documentclass[8pt]{article}
\usepackage[utf8]{inputenc}
\usepackage{geometry}
\geometry{
	letterpaper,
	top=10mm,
	bottom=20mm,
	left=10mm,
	right=10mm,
}
\usepackage{sectsty}
\usepackage{enumitem}

\allsectionsfont{\normalfont\bfseries}
\setlist[enumerate,1]{label=\arabic*.}
\setlist[itemize,1]{label=\textbullet}

\begin{document}
	
	\begin{center}
		\textbf{\Huge{Constitution}}\\\normalsize{of the}\\McMaster Astronomy and Physics Graduate Student Association\\~\\
		published \today
	\end{center}
	\tableofcontents	
	\section*{Abbreviations and Terminology}
	\begin{itemize}
	\item MAPSA -- McMaster Astronomy and Physics (Graduate) Students Association
	\item Standing Member -- Graduate students in the Department of Physics and Astronomy (membership automatic from date of program start until graduation as indicated by congratulatory email by the Graduate Chair) 
	\item academic year -- Sept 1 to Aug 30
	\item calendar year -- Jan 1 to Dec 31
	\item `unanimous approval' -- All Representatives must Approve (motion is without Abstain or Reject)
	\item `majority approval' -- There are more Approves than Rejects by the Representatives (ties dictate failure of the motion, Abstains counts to neither Approve or Reject)
	\item `without disapproval' -- No Representative has Rejected (motion is only with Approves or Abstains)
	\item `by rank' -- Work down the Executive rank: first approach the Chairperson, if they are unavailable then approach the Communication or Treasury Officer, if they are unavailable then approach the Representatives, and so on (outlined by Subsection \ref{subsec:structure})
\end{itemize}
	\pagebreak
	This document is the go-to for all foundational questions regarding MAPSA and its operations. On the chance MAPSA fails due to lack of participation, interest, or support from its members; this document shall serve to outline its mission and regulations to those seeking to revive it. This Constitution is the second, with first MAPSA order (created in 2020) having fallen in 2025.
	\section{Name \& Purpose}
	\subsection{Name}
	The organization of McMaster physics and astronomy graduate students shall hereafter be referred to as the \textbf{McMaster Astronomy and Physics Graduate Student Association}, abbreviated to MAPGSA (``map-G-S-A''), but commonly nicknamed and referred henceforth as \textbf{MAPSA} (``map-sah'').
	\subsection{Purpose}
	MAPSA is a collection of peers among the Department of Physics and Astronomy graduate student body \underline{volunteering} to improve the quality of life of its members by the following purposes:
	\begin{itemize}
		\item To provide a platform for graduate student concerns to be expressed to the department and university and to represent the interests of graduate students at department meetings.
		\item To disseminate important information to the graduate student body.
		\item To coordinate, promote, and enhance graduate student social events.
		\item To assist graduate students in obtaining support for issues related to harassment, discrimination, mental health, and inter-personal conflicts.
	\end{itemize}
	It is important to emphasize that this association must not take away from the personal and academic growth of participating members. Members participating in executive roles in MAPSA must be able and willing -- there shall be no coercion of participation from peers, faculty, or likewise.
	
	\section{Membership}
	\subsection{Eligibility}
	Membership in MAPSA shall be automatic and only for all graduate students in the Department of Physics and Astronomy at McMaster. Such members are Standing Members and remain such between the beginning date of their program until their graduation date (indicated by the congratulatory email by the Graduate Chair). Standing Members are eligible to vote and participate in MAPSA regulatory matters, but participation in MAPSA events is open to all, in accordance with all criteria enumerated in the ``GSA Clubs Operating Policies'' document and is offered irrespective of race, religion, gender, class, age, nationality, ability, marital status, veteran status, or sexual orientation. 
	
	\subsection{Dues}
	There shall be no dues for participating in MAPSA.
	
	\section{Association Structure and Executives}
	\subsection{Structure}\label{subsec:structure}
	Standing members are to be part of Constituencies, categorized by their specialty in each field of physics and astronomy according to their supervisors' specialty. Standing members vote at the beginning of each academic year for their Constituency's Representative. The Representatives then goes on to nominate and elect Officers and Coordinators among the Standing Members (including the Representatives themselves). Officers hold the executive roles in MAPSA, but must seek majority approval from the Representatives for most motions. 
	
	\noindent The rank hierarchy is (at the top) The Chairperson, remaining Officers, Representatives, Coordinators, Meeting Representatives, and then Standing Members. This order is stated strictly for effective MAPSA operations and must not be abused. Those found or accused of flaunting their position are considered a serious offender against MAPSA's core mission and purpose. Said offenders are immediately entered into Dismissal considerations (Section \ref{sec:remove}) without petition.
	\subsection{Constituencies}
	\begin{itemize}
		\item \textbf{Astronomy}: e.g. students under supervision of A. Chen, L. Parker, R. Pudritz, A. Sills, J. Wadsley, C. Wilson, B. Harris, H. Couchman, D. Welch.
		\item \textbf{Theoretical Physics} (Condensed Matter \& High-Energy): e.g. students under supervision of C. Burgess, S. Sibiryakov, D. O'Dell, H. Kunduri, S. Lee, E. Sørensen.
		\item \textbf{Soft Matter and Biophysics}: e.g. students under supervision of K. Dalnoki-Veress, C. Fradin, P. Higgs, M. Rheinstadter, A. Shi.
		\item \textbf{Experimental Condensed Matter}: e.g. students under supervision of B. Gaulin, T. Imai, G. Luke, D. Venus.
		\item \textbf{Rad-Grad}: e.g. students under supervision of F. McNeill, D. Chettle, S. H. Byun, J. Juhasz
		\item \textbf{International Students}: Non-permanent-residents or non-citizens of Canada.
	\end{itemize}
	
	\subsection{Representatives (Elected by Members)}
	Those seeking the role of the Representative of their constituency should first address peers of their constituency before conveying their nomination to the Election Officer. 
	\begin{itemize}
		\item \textbf{Representatives}: To be nominated and voted upon by their corresponding Constituency. There shall be one Representative from each Constituency. Representatives hold approval and veto power over the Officers.
		\item \textbf{Meeting Representative}: To be nominated and voted upon by the Standing Members. Meeting Representatives do not have voting power over the Officers as opposed to Representatives. There shall be two Meeting Representatives elected from the Standing Members.
	\end{itemize}
	\subsection{Officers (Elected by Representatives)}
	The following officers are to be nominated and voted into their positions by the Representatives. Such officers may be nominated persons among the Representatives or any standing member. No one person shall conduct the role of multiple officers. Officers must be in year 1, 2, or 3 of their Ph.D.~program in good academic standing.$^\Gamma$
	\begin{itemize}
		\item \textbf{Chairperson (Officer)}
		\item \textbf{Communications Officer}
		\item \textbf{Treasury Officer$^\star$}
		\item \textbf{Visitor Coordinator(s)$^*$}
		\item \textbf{Events Coordinator(s)$^*$}
		\item \textbf{Election Tsar$^\dagger$}
	\end{itemize}
	$^\Gamma$this is the Continuation Clause. Past collapses of MAPSA has been accredited to high ranking members graduating and not insuring a smooth transition of power and carry-on of traditions and purpose.
	$^\star$the Treasury Officer position shall only be available when MAPSA is approved as a GSA club or when there is otherwise financial reporting that is required for the procurement of finances\\
	$^*$a maximum of two such coordinators may take each position at any time\\
	$^\dagger$this person holds this title from their appointment to the conclusion of the Election
	\section{Responsibilities and Expectations}
	\subsection{Standing Members}
	\begin{itemize}
		\item MAPSA standing members are \underline{expected} to vote for their corresponding Representative and voice dejection of the Executives when any arises. 
		\item Standing members (and participants) at MAPSA-sanctioned events are also \underline{expected} to conduct themselves in manner befitting of Department and University interests--that is to treat all (participants and bystanders) with a high level of mutual respect, professionalism, and courtesy. 
		\item Otherwise, members are encouraged to enjoy their academic journey with this group of friends--MAPSA is founded on attempting to support and enhance this notion.
	\end{itemize}
	\subsection{Representatives}
	\begin{itemize}
		\item
	\end{itemize}
	\subsection{Meeting Representatives}
	\begin{itemize}
		\item Meeting Representatives are \underline{responsible} for attendance at meetings upon the Department's request as ``Graduate Student Representation'' or likewise.\\
		This includes but is not limited to: Department Meetings, Department EDI Committee Meetings, Department Colloquium Meetings. Typically, two Meeting Representatives are preferred at these meetings (as requested by the Department), but one may suffice on instances of urgent absence. 
		\item Upon realization of both Meeting Representatives being unavailable for a Meeting, it is the \underline{responsibility} of the Meeting Representative(s) to seek an able and willing Representative or Officer to attend in their place. Only if the approached Representatives and Officers are also unavailable should the Meeting Representative approach a trusted Standing Member in requesting their aid. The Meeting Representative should report this temporary replacement to the MAPSA Chairperson and the appropriate Department meeting contact.
		\item The Meeting Representatives are \underline{responsible} to convey useful and approved information gleaned at attended meetings to the Standing Members in a concise and accurate fashion. Intelligence gained from the meetings are not all relevant to graduate students and/or may need to remain confidential at the request of the Department. The Meeting Representatives are \underline{expected} to compare notes with each other, draft a concise and easy-to-understand report, gain approval of its publication by the respective Department meeting contact, and only then disseminate the meeting notes.
		\item These meeting notes are \underline{expected} to be disseminated within two calendar weeks of the meeting.
	\end{itemize} 
	\subsection{Chairperson}
	\begin{itemize}
		\item The Chairperson is the head of the association and therefore bears \underline{responsibility} of being the first point-of-contact for any external bodies seeking communication with the P\&A student body. They are \underline{expected} to then relay the curious souls to the appropriate directories. 
		\item They shall be held \underline{responsible} for ensuring MAPSA activities, events, and demeanor are befitting of its mission and purpose. 
		\item The Chairperson takes \underline{responsibility} for the abiding of the procedures, regulations, and traditions of MAPSA as set out by this Constitution and its past members.
		The Chairperson is \underline{responsible} for the organization and execution of duties outlined in Subsections \ref{subsec:app}, \ref{subsec:stateOfTheMAPSA}, and \ref{subsec:orient}.
		\item Upon their appointment, the new Chairperson takes \underline{responsibility} for the continuation MAPSA at the end of the academic year. That is, the Chairperson is \underline{expected} to consider their own academic standing (and that of their fellow Officers) to then motion for a Deputy Chairperson (or suggest the appointment of other Deputy Officers) where applicable.
	\end{itemize}
	\subsection{Communications Officer}
	\subsection{Treasury Officer}
	\subsection{Visitor Coordinator}
	\subsection{Events Coordinator}
	\subsection{Election Tsar}
	\begin{itemize}
	\item an (willing) Election Tsar is to be appointed by the Chairperson \textit{without disapproval} from any Representative before the Address (Subsection \ref{subsec:stateOfTheMAPSA})
	\item 
	\end{itemize}
	\section{Election}\label{sec:elections}
	Elections are to be held each year during the final week of September by all members for their respective constituency's Representatives. Nominations for the elected Representative and Meeting Representative positions will take place in the week prior to elections. Only Standing Members within a Constituency may vote for the Representative of that constituency, while all Standing Members shall vote on nominees for the Meeting Representatives. 
	\begin{itemize}	
		\item Elections are to be held in the ``Ranked Choice Voting'' format and must guarantee the anonymity of the voters. 
		\item Newly elected Representatives take office from the end of the Election to October 1\textsuperscript{st} of the following year (or end of the following Election), unless dismissed.
		\item The Representatives must conclude on a list of Officers before the end of October of each year.
	\end{itemize}
	
	\section{Communication and Meetings}
	\subsection{Communication with Constituents}
	It shall be \textbf{\textit{\underline{mandated}}} that MAPSA is allowed two mass-emails (to all standing members) per year, both in the month of September. The first (from the Chairperson) is to report to all standing members of the previous year's achievements, promote MAPSA and its values to newcoming graduate students, and to announce the imminent election. The second (from the Election Tsar) is to distribute means of voting in the Election. \textit{Unanimous approval} of the Representatives must be obtained for any more mass-emails to be sent to the Standing Members on behalf of MAPSA.
	
	Representatives and Officers are encouraged to promote events \& MAPSA agenda by other means, such as the large Discord Community `Astronomy and Physics Coterie' (as of Sept 2025), bulletin boards on the 2nd/3rd floor of ABB \& in the office space in GSB \& TAB, word-of-mouth, and the dedicated Teams channel.
	
	\subsection{Meetings}
	
	\subsection{Yearly Review, P\&A Grad Student Orientation}
	A review of the Council positions shall be conducted prior to the September election period. The standing representation must be approved by unanimous consent of the Council. In the case of non-unanimous consent, the standing representation may be modified according to the following guidelines:
	\begin{itemize}
		\item If the constituencies are deemed to have failed to properly represent the graduate student body, the Council must present at least two proposals of constituencies, with one necessarily being the standing representation, for referendum. The result of the referendum will amend article 4.2.
		\item If a committee is deemed to have failed to meet the requirements as outlined in article 3.3, or in the case of a new committee seeking representation on the Council, a referendum will be held to amend article 4.3.
	\end{itemize}
	\section{Changes to the Constitution}
	Due to the experimental nature of this MAPSA attempt, changes to the Constitution are allowed by \textit{majority approval} of the Representatives before September 2030. Thereafter, \textit{unanimous approval} of the Representatives is required for changes to the Constitution. Once MAPSA becomes a GSA club, changes to the constitution requires $\geq 60\%$ of its Standing Members' approval.
	
	\section{Removal and Replacement of Representatives and Officers}\label{sec:remove}
	Any Representative, Officer, or Coordinator may seek resignation of their position at any time by notifying the Chairperson. The Chairperson may resign by notifying the Representatives. The Election Tsar is ineligible for resignation and shall understand this burden when agreeing to the position.
	\subsection{Dismissal of Representatives} Standing Members of Constituencies may call for their Representative's removal by petition of more than one-third of its constituents. Such calls are brought to any member of the Representatives or Officers where it will be brought to the Chairperson's attention. The Representative is then dismissed of their duties. Dismissed Representatives who are also an Officer may continue to serve as Officer if agreed \textit{without disapproval} upon by the Representatives (after the replacement of the vacant Representative role).
	\subsection{Dismissal of Officers} Officers are dismissed by disapproving petition of one-third from each constituency or by \textit{majority (dis)approval} of the Representatives. Dismissed Officers who are Representatives remain Representatives unless dismissed by their constituency.
	\subsection{Replacement of Representatives and Officers}
	In the case that multiple roles are vacant, the following is to be completed sequentially. Otherwise, only the procedure(s) relevant to the vacant position need to be followed. 
	\begin{itemize}
		\item Open Representative positions are to be nominated elected by their constituency within two calendar weeks. An Election Tsar is to be appointed by the Chairperson (or by the Representatives \textit{without disapproval}) to conduct these by-elections. By virtue of the regulations, Election Tsar(s) of the by-election(s) may be Representatives or Officers themselves. By-elections are to follow regulations of Elections (Section \ref{sec:elections}).
		\item Open Officer positions are to be nominated and elected by a \textit{full court} of Representatives (that is all Constituencies must have Representation on this decision) within two calendar weeks after the establishment of a \textit{full court}.
		\item Open Coordinator and Meeting Representative positions' roles are to be fulfilled by willing Representatives or Officers until replacements are found. Meeting Representatives are elected by \textit{majority approval} of the Representatives when such vacant positions appear outside of the typical September Election period.
	\end{itemize}
\end{document}